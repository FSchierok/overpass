\section{Executes}
\subsection{Buyrounds}
\subsubsection{Default}
\begin{figure}
    \centering
    \includegraphics[width=0.8\textwidth]{img/default.png}
    \caption{Default Aufbau.}
    \label{fig:default}
\end{figure}
Der Default legt den Fokus auf B-Kurz. Um das einzunehmen wird der Kurz-Molotov \ref{fig:kurz_molly} und vlt von Schwarz die Kurz Flash \ref{fig:kurz_flash} geworfen.
Gelb bleibt im Connector hinten, bis rot in den Connector reinflasht und sichert dann gemeinsam mit Blau den Connector oben. 
Grün hällt erstmal Brunnen und A-Lang, und kann dann mit Gelb zusammen Party einnehmen.
\begin{figure}
    \centering
    \includegraphics[width=0.8\textwidth]{img/default2.png}
    \label{fig:default2}
\end{figure}
Der alternative Default legt den Fokus auf A. Schwarz hällt B defensiv. Gelb und Grün gehen nach A Lang. Hier kann je nach CT Setup geflasht werden. Gelb muss auf dem Weg die Toilets Smoke \ref{fig:toilets} smoken.
Mit dieser Smoke kann Rot auf Party vorgehen und zusammen mit Blau den Connector sichern.
\FloatBarrier
\subsubsection{Execute A}
\begin{figure}
    \centering
    \includegraphics[width=0.8\textwidth]{img/exec_a.png}
    \caption{Execute auf A.}
    \label{fig:exec_a}
\end{figure}
Grün und Gelb gehen zusammen nach Lang und werfen die GSG Smoke \ref{fig:gsg_smoke}. Blau und Rot gehen durch Toilets bis Sandwich und teilen sich da auf. 
Rot molliet Truck \ref{fig:truck_molly}. Schwarz sichert Kurz bis Rot den Molly geworfen hat, dann wirft er die Truck/Spot Smoke \ref{fig:truckspot} und hällt danach Connector.
\FloatBarrier
\subsubsection{Execute B}
\begin{figure}
    \centering
    \includegraphics[width=0.8\textwidth]{img/exec_b.png}
    \caption{Execute auf B.}
    \label{fig:exec_b}
\end{figure}
Rot smoket Bridge \ref{fig:bridge_kurz} und Blau molliet Toxic \ref{fig:toxic_molly}. Danach gehen beide über Kurz auf den Spot. 
Schwarz smoket den Spot \ref{fig:spot_smoke} und geht dann mit Gelb über Monster. Grün hällt Connector. von Party oder Connector aus. Der Planter muss gegen Wasser gedeckt werden.
 Nach dem Plant sollte Wasser und am Besten noch Galaxy eingenommen werden.
 \FloatBarrier
\subsection{AntiEco}
\begin{figure}
    \centering
    \includegraphics[width=0.8\textwidth]{img/antieco_setup.png}
    \caption{Das Setup gegen Ecos.}
    \label{fig:antieco_setup}
\end{figure}
Das AntiEco Setup ist deutlich defensiver. Schwarz geht Über Spielplatz und sichert A-Lang. Grün hällt von T-Stairs Brunnen und kann dort naden. Gelb hällt den Connector, am Besten von links, sodass er über die Leiter zurückfalen kann.
Rot hällt vom T-Spawn Monster und Blau schaut durch die Kurzröhre B-Kurz.
\subsubsection{AntiEco A}
\begin{figure}
    \centering
    \includegraphics[width=0.8\textwidth]{img/antieco_a.png}
    \caption{AntiEco Execute auf A.}
    \label{fig:antieco_a}
\end{figure}
Bei der Antieco A gehen alle geschlossen über A Lang. Grün molliet Truck \ref{fig:truck_molly} und smoket Fence \ref{fig:fence_smoke_lang}. 
Danach bleibt er auf Lang Rot sichert dass alle anderen Hinter die Site können. Blau deckt den Planter und Schwarz nimmt Bank ein. Es sollte für Bank gelegt werden \ref{fig:plant_a_Bank}.
\FloatBarrier
\subsubsection{AntiEco B}
\begin{figure}
    \centering
    \includegraphics[width=0.8\textwidth]{img/antieco_b.png}
    \caption{Antieco Execute für B}
    \label{fig:antieco_b}
\end{figure}
Bei dem Antieco B Execute gehen alle geschlossen über Lang. Es wird die Glave Smoke geworfen \ref{fig:glave}. Gelb geht nach Kurz und molliet dort. Blau und Rot gehen nach CT/Galaxy vor und mollien dabei CT.
Grün deckt Wasser. Schwarz legt die Bombe, am Besten für Heaven \ref{fig:plant_b_heaven}. Blau kann auch nach Heaven durch gehen, muss dabei aber vorsichtig sein.